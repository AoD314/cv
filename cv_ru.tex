\documentclass[12pt,a4paper]{moderncv}

\moderncvtheme[purple]{classic}
\usepackage[utf8]{inputenc}
\usepackage[unicode]{hyperref}
\usepackage[english,russian]{babel}
\usepackage{amsmath,amsthm,amssymb}
\usepackage{mathtext}
\usepackage[T1,T2A]{fontenc}


\usepackage{geometry}
\geometry{left=2.0cm}
\geometry{right=1.5cm}
\geometry{top=1.0cm}
\geometry{bottom=1.5cm}

\firstname{Андрей}
\familyname{Морозов}
\address{}{Нижний Новгород, Россия}
\mobile{+7 920 065 27 75}
\email{morozov.andrey.vmk@gmail.com}
\homepage{https://github.com/aod314/}

\makeatletter
\renewcommand*{\bibliographyitemlabel}{\@biblabel{\arabic{enumiv}}}
\makeatother

\moderncvcolor{green}

\begin{document}
\maketitle

\section{О себе}
Cпециалист в области тестирования, автоматизация и оптимизация технических процессов сборки и развертывания программных систем,
а так же в области алгоритмов компьютерного зрения, аппаратной архитектуры и управления проектами.

\section{Опыт}
\cventry{\textbf{Октябрь 2022 - Текущий момент}}
    {Теком}{Нижний Новгород, Россия}{}{}
    {Senior Software Engineer}

\cventry{\textbf{Ноябрь 2020 - Август 2022}}
    {NVIDIA}{Нижний Новгород, Россия}{}{}
    {DevOps инженер}

\cventry{}
    {Оптимизация}{Ноябрь 2020 - Август 2022}{}{}
    {Оптимизация процессов CI, Развертывание на различных операционных системах и архитектурах.}

\cventry{\textbf{Июнь 2017 - Ноябрь 2020}}
    {Aquantia/Marvell Semiconductor}{Нижний Новгород, Россия}{}{}
    {QA инженер}

\cventry{}
    {Автотесты}{2017-2018}{}{}
    {Написание автоматических тестов и мультиплатформенные скрипты автоматизации.}

\cventry{}
    {Тестирование PHY}{2018-2019}{}{}
    {Написание различных тестов для PHY.}

\cventry{}
    {CI/CD}{2019-2020}{}{}
    {Поддержка и улучшения существующей CI/CD системы.}

\cventry{\textbf{Ноябрь 2015 - Июнь 2017}}
    {АНКХ-НН}{Нижний Новгород, Россия}{}{}
    {Генеральный директор}

\cventry{}
    {HighLoad}{Сентябрь 2016 - Июнь 2017}{}{}
    {Разработка и реализация веб приложения для высоканагруженных систем.}

\cventry{\textbf{Сентябрь 2009 - Апрель 2016}}
    {Itseez}{Нижний Новгород, Россия}{}{}
    {Инженер по програмному обеспечению}

\cventry{}
    {CI}{2010-2011}{}{}
    {Развертывание CI системы Buildbot для проекта OpenCV \newline(\url{http://build.opencv.org/buildbot})}

\cventry{}
    {Строительство}{Май 2012 - Сентябрь 2012}{Камчия, Болгария}{}
    {Надзор по строительству, монтаж, пуско-наладочные работы цифровой системы планетария.}

\cventry{}
    {Agile}{2013 - 2015}{}{}
    {Соавтор и лектор курсов по гибким методологиям разработки в ННГУ.\newline(\url{https://github.com/unn-vmk-software/agile-course-theory})}

\cventry{}
    {2D/3D компьютерное зрение}{Сентябрь 2012 - Январь 2014}{}{}
    {Калибровка RGB камер для построения карты глубины. Создание приложений для Android. Управление командой.}

\cventry{}
    {Оптимизация}{Январь 2014 - Апрель 2016}{}{}
    {Низкоуровневая оптимизация алгоритмов компьютерного зрения в бибилиотке OpenCV для специальной архитектуры(DSP)}

\cventry{\textbf{Июнь 2005 - Май 2010}}
    {Школа \#91}{Нижний Новгород, Россия}{}{}
    {Педагог дополнительного образования. Адмитистратор компьютерного класса.}

\section{Образование}
  \cventry
    {2005 - 2011}
    {Университет Лобачевского, ННГУ}
    {Нижний Новгород, Россия}
    {}{}
    {Факультет прикладной математики и кибернетики. Кафедра компьютерного программного обеспечения.\newline{}
    Степень магистра}


\section{Основные навыки}
\cvline  {Языки программирования}{python, bash, c/c++}
\cvline  {Операционные системы}{Linux, Windows, macOS}
\cvline  {Техники разработки ПО}{\textbf{Agile:} TDD, Continuous Integration, XP, Scrum, Lean, Kanban}
\cvline  {Системы контроля версий}{Git}
\cvline  {Инструменты}{docker, k8s, jenkins, cmake, ninja, gcc/clang, vim}
\cvline  {Библиотеки}{opencv, tbb, cuda, opencl, pytest, postgresql}

\section{Редкоиспользуемые навыки}
\cvline {Языки программирования}{\LaTeX, Go, Rust}

\section{Интересы}
\cvline { }{DevOps, CI/CD, Optimization, HighLoad, QA, BigData}


\section{Вклад в проекты с открытым исходным кодом}
\cvline
  {opencv}
  {\url{https://github.com/itseez/opencv}\newline{}
  Библиотека компьютерного зрения (\url{http://opencv.org})}

\cvline
  {mpv}
  {\url{https://github.com/mpv-player/mpv}\newline{}
  Видео плеер mpv основанный на кодовой базе MPlayer/mplayer2 \newline (\url{http://mpv.io})}

\cvline
  {agile}
  {
  \url{https://github.com/unn-vmk-software/agile-course-theory}\newline{}
  \url{https://github.com/unn-vmk-software/agile-course-practice}\newline{}
  Основной курс по гибким методологиям разработки (теория и практика)
  }

\end{document}
